\documentclass{dcbl/challenge}

\setdoctitle{Collaborate using GitHub}
\setdocauthor{Stephan Bökelmann}
\setdocemail{sboekelmann@ep1.rub.de}
\setdocinstitute{AG Physik der Hadronen und Kerne}


\begin{document}

GitHub and similar platforms offer useful features not only for storing code, but also for sharing it.
To collaborate effectively on GitHub, you need to know a few rules.
Although there are several modes of collaborating, the one explained here is probably the most general and effective.
Our goal is to work with multiple people on the same project, therefore you should work with at least one partner in this exercise.

\section*{Exercises}
\begin{aufgabe}
    Go to \url{https://github.com/} and create a new repository.
    Initialize this repository with a \texttt{README.md} file.
    Clone this repository to your computer, change the \texttt{README.md} file and push the changes to your repository.
    This should be done by all partners in this exercise.
\end{aufgabe}

\begin{aufgabe}
   Begin by sharing the links to your GitHub repositories with your collaboration partners. This will allow each of you to access and view each other's projects. While directly cloning the repository is a straightforward method to start working with the project, there's a more collaborative approach that leverages GitHub's features.
    
    Notice the \textbf{Fork} button located at the top right corner of the repository page? Clicking on this button will create a copy of the repository under your own GitHub account. This process, known as forking, grants you your own version of the project to work on independently. Each member of the group should fork another partner's repository, ensuring that everyone has the opportunity to contribute to a different project.
    
    Forking is essential in scenarios where direct contributions to the original repository are restricted due to permissions—typically, you don't have the right to push changes to someone else's project. However, public repositories can always be forked, allowing you to make your modifications and subsequently suggest these changes to the original project. This suggestion is done through a pull request, which the original repository's owners can review. For this process to work effectively, ensure that your fork is set to be publicly accessible, allowing others to see and evaluate your contributions.
\end{aufgabe}

\begin{aufgabe}
    After successfully forking a partner's repository to your own GitHub account, the next step involves cloning this forked repository to your local machine. This enables you to work on the project directly from your own development environment. To clone the repository, use the git clone command followed by the URL of your forked repository. You can find this URL on your fork's GitHub page.
    
    Make some changes to the \texttt{README.md} file. Next, it's time to commit them and push the updates back to your repository on GitHub. Use the \texttt{git add README.md} command to stage your changes, followed by  \texttt{git commit -m} to commit them with a meaningful message. Finally, push your changes using \texttt{git push}. 
    
    After pushing the changes, inspect the GitHub-webpage of your own fork.
\end{aufgabe}

\begin{aufgabe}
    With the changes you've made now present in your fork, it's time to consider contributing them back to the original project. This step is taken when you believe your modifications could benefit the project as a whole. To initiate this process, we want to create a Pull request that your partner can decide on.
    
    Your fork has two additional buttons: A \textbf{contribute} and a \textbf{sync} button. With the \textbf{contribute}-Button, you get a setup for a Pull Request that chooses the main branch on your partner's repo as the target branch.
    
    Use this page, to tell the partner about the recent changes, and why you believe, that this change is relevant. This involves writing a description of your changes and why they should consider accepting your PR. Though this feels a little over-the-top right now, this is one of the most important parts of collaboration.
    
    When you are finished, your partner will get a notification about the Pull request in their repository. 
\end{aufgabe}

\begin{aufgabe}
    Go back to your own repository which your partners forked. You should have received their pull-requests at this point.
    
    To see them, navigate to the pull-request view of your repository via the pull-request tab on the top. 

    Select you partner's pull-request and review it. There are multiple options to request changes, review commits and have discussions about the contribution.
    
    Finally, if you are fine with them, merge them into your own repository!
\end{aufgabe}

\section*{Annotations}
\begin{enumerate}
    \item Link zu einem YouTube-Video: \url{https://www.youtube.com}
\end{enumerate}

\end{document}
