\documentclass{dcbl/challenge}

\setdoctitle{Git Remote}
\setdocauthor{Stephan Bökelmann}
\setdocemail{sboekelmann@ep1.rub.de}
\setdocinstitute{AG Physik der Hadronen und Kerne}
\usepackage{listings}

\begin{document}

Working with Git not only enables us to work with multiple savepoints, but also allows us to work with multiple collaborators.
Sharing savepoints - commits - between group-members is possible with the help of Git Remotes.
A repository is a collection of files, typically organized in one directory, in which also a \texttt{.git} folder is present.
A remote repository the same, but on the filesystem of another machine.
We have to keep in mind, that there is no such thing as \textit{The Internet}, but just other peoples computers.
Networks are basically distributed filesystems.
Each file in the network has a unique name, just as the files in our own filesystem.
Instead of just having a \texttt{/home/user/} folder, we have a \texttt{IP:filepath} for a resource inside of the network.
This is also called a URI - Uniform Resource Identifier.
In naive networks, we can store and load files, just as we would in our own filesystem.
To prevent malicious actors from tinkering with other peoples computers a concept called server was introduced. 
Instead of accessing a remote computers filesystem directly, the remote computer is running a program, which has the responsibility to make sure, that only the right person can access only a limited subset of the remote filesystem.
The most common type of server might be a webserver, whichs purpose is to provide access to files in a specific directory.
Git Remotes work in sort of a similar way.
We can read and write files from and to a remote repository, a repository, which is stored on the filesystem of another machine.
A webserver or an SSH-server can be used to provide access to the remote repository.

\section*{Exercise}
\begin{aufgabe}
    Originally, Git was supposed to synchronize the filesystems of two or more co-workers inside of a trusted network.
    To work with multiple collaborators, we need to know, where they stored their repository. 
    In the following examples, we will use an online-platform, called GitHub, as an example.
    Just think of GitHub as a remote repository with an additional website to manage the repository.
    There are three different commands we'd like to explore now:
    \begin{enumerate}
        \item \texttt{git clone}
        \item \texttt{git pull}
        \item \texttt{git push}
    \end{enumerate}
    Let's assume, that you just started working at a new company.
    Your filesystem is empty, but one of your co-workers has already been working on a project.
    You decide, that you'd like to chime in and work with your co-worker.
    To get an exact copy of your co-workers repository, you need to use \texttt{git clone}.
    This not only copies the current status of the repository, but also the history of the repository.
    Now you both work on the same project. 
    Some day, your co-worker says, that he made major changes to the project and you should pull them into your local repository.
    Not to get ahead of ourselves here, but you may already keep the term \texttt{pull-request} in your mind, for somebody asking you, to pull their new code into your repository.
    Since you just want to get the newest changes and don't want to copy all of the history, you can use \texttt{git pull}.
    \footnote[1]{\texttt{git pull} is also known as \texttt{git fetch} followed by \texttt{git merge}. But that should bother us later.}
    Let's assume, that your co-worker did some really neccessary changes and you shouldn't be able to run the old code any longer.
    If there is a certain level of trust between you and your co-worker, they may use \texttt{git push}.
    By using \texttt{git push}, changes can be send to another repository. 
    The other repository, that is not on our own computer is called a remote repository. 
    We can add a link to a remote repository to our local repository by using \texttt{git remote add}.
    Go to GitHub and create a new \textbf{empty} repository.
    Navigate into your local repository and use \texttt{git remote add myRemoteRepo git@github.com:<your-username>/<your-repository-name>.git} to link your local repository to your remote repository.
    Run \texttt{git push myRemoteRepo} to push your local repository to your remote repository.
    This command should fail!
\end{aufgabe}

\begin{aufgabe}
    To be able to push to a remote repository, we are using the SSH-protocol.
    As you shall already know, we should use a keypair to authenticate ourselves to the remote repositories server.
    Thus we need to generate a keypair, or use an existing one.
    Copy your public key and add it to your GitHub-Account.
    Go to \href{https://github.com/settings/keys}{your profile-settings}, and click on \texttt{New SSH key}.
    Add your key to the authorized keys.
    Git doesn't allow us to give an identity-file as a parameter to \texttt{git push}.
    But that is no problem, we will just use our \texttt{$\sim$/.shh/config}-file instead and define the key that should be used in there.
    If the file doesn't already exist, you can create it.
    Add the follwowing lines to your \texttt{$\sim$/.shh/config} file:
    \begin{lstlisting}
    Host github.com
        HostName github.com
        IdentityFile <path-to-your-private-key>
    \end{lstlisting}
    This config makes sure, that Git will use our private key to authenticate itself to the remote repositories server.
    Run \texttt{git push myRemoteRepo} again.
\end{aufgabe}

\begin{aufgabe}
    Delete your local repository and and use the \texttt{git clone} command, to get an exact copy of your remote repository.
    Add some files, and push them again. 
\end{aufgabe}

\begin{aufgabe}
    Create a new branch, change some files, and push it to your remote repository.
    Describe what you can see on the GitHub page of your remote repository.
\end{aufgabe}

\begin{aufgabe}
    Try to merge something in your own repo
\end{aufgabe}

\section*{Anmerkungen}
\begin{enumerate}
    \item Link zu einem YouTube-Video: \url{https://www.youtube.com}
\end{enumerate}

\end{document}
